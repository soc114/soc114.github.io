
\documentclass{beamer}
\usecolortheme{dove}
\setbeamertemplate{navigation symbols}{}
\usepackage{amsmath,amssymb,amsfonts,amsthm, multicol, subfigure, color}
\usepackage{bm}
\usepackage{graphicx}
\usepackage{tabularx}
\usepackage{booktabs}
\usepackage{hyperref}
\usepackage{pdfpages}
\usepackage{xcolor}
\definecolor{seagreen}{RGB}{46, 139, 87}
\definecolor{mustard}{RGB}{234, 170, 0}
\def\independenT#1#2{\mathrel{\rlap{$#1#2$}\mkern2mu{#1#2}}}
\newcommand\indep{\protect\mathpalette{\protect\independenT}{\perp}}
\def\log{\text{log}}
\newcommand\logit{\text{logit}}
\newcommand\iid{\stackrel{\text{iid}}{\sim}}
\newcommand\E{\text{E}}
\newcommand\V{\text{V}}
\renewcommand\P{\text{P}}
\newcommand{\Cov}{\text{Cov}}
\newcommand{\Cor}{\text{Cor}}
\newcommand\doop{\texttt{do}}
\usepackage{stackrel}
\usepackage{tikz}
\usetikzlibrary{arrows,shapes.arrows,positioning,shapes,patterns,calc}
\newcommand\slideref[1]{\vskip .1cm \tiny \textcolor{gray}{{#1}}}
\newcommand\red[1]{\color{red}#1}
\newcommand\blue[1]{\color{blue}#1}
\newcommand\gray[1]{\color{gray}#1}
\newcommand\seagreen[1]{\color{seagreen}#1}
\newcommand\purple[1]{\color{purple}#1}
\newcommand\orange[1]{\color{orange}#1}
\newcommand\black[1]{\color{black}#1}
\newcommand\white[1]{\color{white}#1}
\newcommand\teal[1]{\color{teal}#1}
\newcommand\magenta[1]{\color{magenta}#1}
\newcommand\Fuchsia[1]{\color{Fuchsia}#1}
\newcommand\BlueGreen[1]{\color{BlueGreen}#1}
\newcommand\bblue[1]{\textcolor{blue}{\textbf{#1}}}
\newcommand\bred[1]{\textcolor{red}{\textbf{#1}}}
\newcommand\bgray[1]{\textcolor{gray}{\textbf{#1}}}
\newcommand\bgreen[1]{\textcolor{seagreen}{\textbf{#1}}}
\newcommand\bref[2]{\href{#1}{\color{blue}{#2}}}
\colorlet{lightgray}{gray!40}
\pgfdeclarelayer{bg}    % declare background layer for tikz
\pgfsetlayers{bg,main} % order layers for tikz
\newcommand\mycite[1]{\begin{scriptsize}\textcolor{darkgray}{(#1)}\end{scriptsize}}
\newcommand{\tcframe}{\frame{
%\small{
\only<1|handout:0>{\tableofcontents}
\only<2|handout:1>{\tableofcontents[currentsubsection]}}
%}
}

\newcommand{\goalsframe}{\begin{frame}{Learning goals for this course}
By the end of this course, you will be able to
\begin{itemize}
\item connect theories about inequality\\to quantitative empirical evidence
\item evaluate the effects of hypothetical\\interventions to reduce inequality
\item conduct data analysis using the\\R programming language
\end{itemize} \vskip .2in
\end{frame}}

\usepackage[round]{natbib}
\bibliographystyle{humannat-mod}
\setbeamertemplate{enumerate items}[default]
\usepackage{mathtools}

\title{Social Data Science}
\author{Ian Lundberg}
\date{\today}

\begin{document}

\begin{frame}
\begin{tikzpicture}[x = \textwidth, y = \textheight]
\node at (0,0) {};
\node at (1,1) {};
\node[anchor = north west, align = left, font = \huge] at (0,.9) {Social Data Science};
\node[anchor = north east, align = right] (number) at (1,.9) {SOCIOL 114\\Winter 2026};
\node[anchor = north, font = \Large, align = left] at (.5,.5) {\bblue{Lecture 1: Welcome}};
\end{tikzpicture}
\end{frame}

\goalsframe

\begin{frame}{How computing looked \textbf{in the 1950s}}
\begin{tikzpicture}[x = \textwidth, y = .8\textheight]
\node[anchor = north] (fig1) at (0,0) {\includegraphics[width = .8\textwidth]{figures/punch_cards_nasa}};
\node[anchor = north west, font = \footnotesize] at (fig1.south west) {Source: \href{https://twitter.com/NASAhistory/status/1240971354003955712/photo/1}{NASA}};
\end{tikzpicture}
\end{frame}

\begin{frame}{How computing looked \textbf{in the 1980s}}
\begin{tikzpicture}[x = \textwidth, y = .8\textheight]
\node[anchor = north] (fig2) at (0,0) {\includegraphics[width = .8\textwidth]{figures/pc}};
\node[anchor = north west, font = \footnotesize] at (fig2.south west) {Source: \href{https://commons.wikimedia.org/wiki/File:Commodore_PET_Exhibit_at_American_Museum_of_Science_and_Energy_Oak_Ridge_Tennessee.jpg}{Wikimedia}};
\end{tikzpicture}
\end{frame}

\begin{frame}{How computing looks \textbf{today}}
\begin{tikzpicture}[x = \textwidth, y = .8\textheight]
\node[anchor = north] (fig2) at (0,0) {\includegraphics[width = \textwidth]{figures/macbook}};
\node[anchor = north west, font = \footnotesize] at (fig2.south west) {Source: \href{https://www.apple.com/macbook-air-m1/}{Apple}};
\end{tikzpicture}
\end{frame}

\begin{frame}{Social Data Science}

Our class will combine
\begin{itemize}
\item new data science tools for estimation
\item longstanding social science research designs
\end{itemize} \vskip .2in
We will seek to tell stories with data.

\end{frame}


\begin{frame}{Figure from \bref{https://doi.org/10.1073/pnas.1918891117}{England et al. (2020)}}
\includegraphics[height = .8\textheight]{figures/gender_emp.png}
\end{frame}

\begin{frame}{Research questions in social data science}

Which of the following can be answered by data science?
\begin{enumerate}
\item The gender employment gap closed  from 1970--1990 but has remained roughly constant since then.
\item We should enact new policies to eliminate the remaining gap.
\end{enumerate} \vskip .2in \pause
(1) is an \textbf{empirical question} (more amenable to data science) \\
(2) is a \textbf{normative question} (less amenable to data science)
\end{frame}

\begin{frame}{Elements of a data science question} 

\pause

\begin{enumerate}
\item a unit of analysis
\begin{itemize}
\item a row of your dataset \pause
\end{itemize}
\item an outcome
\begin{itemize}
\item a variable with a value for each unit \pause
\end{itemize}
\item a target population
\begin{itemize}
\item a set of units about whom to infer
\item clear who is included and who is not
\end{itemize} 
\end{enumerate}

\end{frame}

\section{Course Logistics}

\section{Course Logistics}

\begin{frame}
\huge Course logistics
\end{frame}

\begin{frame}{Who should take this course?}

The course is designed for upper-division undergraduate students.

\end{frame}

\begin{frame}{Attendance}

Public health matters---stay home if sick! \vskip .2in

Otherwise, we hope to see you in class.

\end{frame}

\begin{frame}{Course materials}

All materials will be posted here:\vskip .2in
\begin{Large}\bref{https://soc114.github.io/}{soc114.github.io}\end{Large}

\end{frame}

\begin{frame}{Course support}

\begin{itemize}
\item Post questions on \bref{https://piazza.com/class/miyskvk64nv4y1}{Piazza}
\item Office hours
\end{itemize}

\end{frame}

\begin{frame}{Software}

As soon as possible, you should
\begin{itemize}
\item \bref{https://cloud.r-project.org/}{Install R} \hfill (statistical software)
\item \bref{http://www.rstudio.com/download}{Install RStudio} \hfill (user interface)
\end{itemize}
There are also cloud-based options.

\end{frame}

\begin{frame}{Method of assessing student achievement}

\begin{tabular}{rl}
Quizzes & 50\% \\
Problem sets & 50\%
\end{tabular}

\end{frame}

\begin{frame}{Quizzes}

\begin{itemize}
\item Due MW at 5pm
\item Submit in BruinLearn after each lecture
\item Automatically graded
\item Attending class will make them easier
\item Lowest 2 scores dropped at end of the quarter
\end{itemize}

\end{frame}

\begin{frame}{Problem sets}

\begin{itemize}
\item Due Friday at 5pm
\item Material covered by Tuesday
\item Graded by PhD student reader
\item Content includes
\begin{itemize}
\item Code to analyze data
\item Written summaries in English
\end{itemize}
\end{itemize}

\end{frame}

\begin{frame}{Late work}

\begin{itemize}
\item 0.5\% penalty for each hour late
\item 1 minute late = 0.5\% penalty
\item 23:01 late = 24 x 0.5 = 12\% penalty
\item Automatic in BruinLearn
\end{itemize}

No assignments will be accepted after Mar 20 at 5pm.

\end{frame}

\begin{frame}{Collaboration}

\begin{itemize}
\item encouraged to work together
\item consulting help is great
\item should never involve one student having possession of a copy of all or part of work done by someone else, in the form of an email, an email attachment file, or a hard copy
\end{itemize}

\end{frame}

\begin{frame}{Academic integrity}

Each student in this course is expected to abide by the UCLA \href{https://deanofstudents.ucla.edu/academic-integrity}{Academic Integrity} policies. Any work submitted by a student in this course for academic credit must be the student’s own work.

\end{frame}

\begin{frame}{Students with disabilities}

You belong in this course. We are happy to work with you on appropriate accomodations---see the syllabus for details about working with CAE.

\end{frame}

\begin{frame}{Mental health and wellbeing} 

Your health and wellbeing are important to us! \vskip .2in

See syllabus for links to mental health resources. We hope our course helps you thrive at UCLA, and your thriving is far more important than anything in this course.

\end{frame}

\begin{frame}{Honors section (Soc 189)}
\begin{itemize}
\item Capped at 20 seats
\item Open to any student in this course!
\item W 1--1:50pm
\item Leads to an independent research report analyzing an empirical question of your choosing
\end{itemize}
\end{frame}

\begin{frame}

Questions about logistics?

\end{frame}

\section{Intro to R}

\begin{frame}{Introduction to R}

Using the course website, we will learn about
\begin{itemize}
\item Software Prerequisites
\item Basics of R
\end{itemize}

\end{frame}

\end{document}


