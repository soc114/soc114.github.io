
% MIGHT MOVE ALL SLIDES TO QMD FOR CODE PURPOSER

\documentclass{beamer}

\usetheme{default}
\usepackage{tikz}
\usetikzlibrary{arrows,shapes.arrows,positioning,shapes}
\usepackage{graphicx}
\usepackage{hyperref}
\newcommand\red[1]{{\color{red}#1}}
\newcommand\bred[1]{{\color{red}\textbf{#1}}}
\newcommand\blue[1]{{\color{blue}#1}}
\newcommand\bblue[1]{{\color{blue}\textbf{#1}}}
\newcommand\green[1]{{\color{olive}#1}}
\newcommand\bgreen[1]{{\color{olive}\textbf{#1}}}
\newcommand\black[1]{{\color{black}#1}}
\newcommand\white[1]{{\color{white}#1}}
\newcommand\E{\text{E}}
\newcommand\V{\text{V}}
\renewcommand\P{\text{P}}
\definecolor{seagreen4}{RGB}{46, 139, 87}


\setbeamertemplate{navigation symbols}{}
%\setbeamertemplate{footline}[text line]{%
%\parbox{\linewidth}{\vspace*{-8pt}Ian Lundberg (UCLA)}}
\addtobeamertemplate{footline}{%
  \leavevmode%
  \hbox{%
  \begin{beamercolorbox}[wd=\paperwidth,ht=2.25ex,dp=1ex,center]{author in head/foot}%
     \insertsectionnavigationhorizontal{\paperwidth}{}{}
  \end{beamercolorbox}}%

}

\newcommand{\indep}{{\bot\negthickspace\negthickspace\bot}}

\newcommand{\goalsframe}{\begin{frame}{Learning goals for today}
By the end of class, you will be able to
\begin{itemize}
    \item understand sample splitting: a common data science procedure for choosing among many candidate estimators
 \end{itemize} 
  \vskip .2in
\end{frame}}

\title{}

\author{}

%\pgfdeclareimage[height=1cm]{university-logo}{ff_logo.jpg}
%\logo{\pgfuseimage{university-logo}}

%%%%%%%%%%%%%%%%%%%%%%%%%%%%%%
\begin{document}

\begin{frame}
\begin{tikzpicture}[x = \textwidth, y = \textheight]
\node at (0,0) {};
\node at (1,1) {};
\node[anchor = north west, align = left, font = \huge] at (0,.9) {Social\\Data\\Science};
\node[anchor = north east, align = right] (number) at (1,.9) {Soc 114\\Winter 2026};
\node[anchor = north, font = \Large, align = center] at (.5,.5) {Data-Driven Estimator Selection};
\end{tikzpicture}
\end{frame}

\begin{frame}{Learning goals}
\begin{itemize}
\item k-nearest-neighbors estimator
\item bias-variance tradeoff
\item sample splitting
\item cross validation
\end{itemize}
\end{frame}

\section{Example}

\begin{frame}{A running example}

% Should define data at team level
\begin{itemize}
\item Sample 10 players from each MLB team
\item Estimate sample average salary on each team
\item Produces data where
\begin{itemize}
\item Unit of analysis $i$ is a team
\item Outcome $y_i$ is average salary
\item Predictor $x_i$ is prior year average salary
\end{itemize}
\end{itemize} \vskip .1in

Goal: Predict mean salary of all Dodgers (sampled and unsampled)

\end{frame}

\begin{frame}{To do: Code so they can generate team estimates}
%\texttt{source("sample_team_estimates.R")}
%\texttt{sample <- sample\_team\_estimates(n = 10)}
\end{frame}

\section{Nearest Neighbors}

\begin{frame}{Task}



\end{frame}

\begin{frame}{Estimator: k-nearest neighbors}

10 sampled players per team
\begin{itemize}
\item Dodger sample mean might be noisy
\item Could pool with similar teams defined by past mean salary
\begin{itemize}
\item Dodgers: 8.39m
\item 1st-nearest neighbor. NY Mets: 8.34m
\item 2nd-nearest neighbor. NY Yankees: 7.60m
\item 3rd-nearest neighbor. Philadelphia: 6.50m
\end{itemize}
\item How does performance change with the number of neighbors included?
\begin{itemize}
\item measured by mean squared prediction error
\end{itemize}
\end{itemize}

\end{frame}

\begin{frame}
\includegraphics<1>[width = \textwidth]{figures/knn_0}
\includegraphics<2>[width = \textwidth]{figures/knn_1}
\includegraphics<3>[width = \textwidth]{figures/knn_2}
\includegraphics<4>[width = \textwidth]{figures/knn_3}
\includegraphics<5>[width = \textwidth]{figures/knn_4}
\includegraphics<6>[width = \textwidth]{figures/knn_5}
\includegraphics<7>[width = \textwidth]{figures/knn_10}
\includegraphics<8>[width = \textwidth]{figures/knn_15}
\includegraphics<9>[width = \textwidth]{figures/knn_20}
\includegraphics<10>[width = \textwidth]{figures/knn_25}
\end{frame}

\begin{frame}
\includegraphics[width = \textwidth]{figures/knn_29}
\end{frame}

\section{Sample Splitting}

\begin{frame}
You have one sample.\\
How do you estimate out-of-sample performance?
\end{frame}

\begin{frame}
\includegraphics[width = \textwidth]{figures/randomize_split}
\end{frame}

\begin{frame}
\includegraphics[width = \textwidth]{figures/train_test}
\end{frame}

\begin{frame}{Exercise: Conduct a sample split in code}

\begin{enumerate}
\item Sample 10 players per team
\item Take a 50-50 sample split stratified by team
\item Fit a linear regression in the train set
\item Predict in the test set
\item Report mean squared error
\end{enumerate}

\end{frame}

\section{Cross Validation}


\begin{frame}{Cross Validation}

A train test split loses lots of data to testing. \vskip .2in
Is there a way to bring it back?

\end{frame}

\begin{frame}{Cross Validation}
\begin{tikzpicture}[x = \textwidth, y = .8\textheight]
\node at (0,0) {};
\node at (1,1) {};
\node[anchor = north west, align = center] at (0,1) {Randomize\\to 5 folds};
\node[anchor = north west] at (0,.9) {\includegraphics[scale = .8]{figures/five_folds}};
\node<2->[anchor = north, align = center] at (.55,1) {Iteratively use each as the test set};
\node<3->[anchor = north west] at (.2,.9) {\includegraphics[scale = .8]{figures/five_folds_1}};
\node<4->[anchor = north west] at (.35,.9) {\includegraphics[scale = .8]{figures/five_folds_2}};
\node<5->[anchor = north west] at (.5,.9) {\includegraphics[scale = .8]{figures/five_folds_3}};
\node<6->[anchor = north west] at (.65,.9) {\includegraphics[scale = .8]{figures/five_folds_4}};
\node<7->[anchor = north west] at (.8,.9) {\includegraphics[scale = .8]{figures/five_folds_5}};
\node<8->[anchor = north, align = center] at (.55,.1) {Average prediction error over folds};
\end{tikzpicture}
\end{frame}

\begin{frame}
Out-of-sample predictive performance is not just for tuning parameters. \vskip .2in
It can help you choose your algorithm.
\end{frame}

\begin{frame}
\includegraphics[width = \textwidth]{figures/many_algorithm_predictions}
\end{frame}

\begin{frame}
\includegraphics[width = \textwidth]{figures/many_algorithm_mse}
\end{frame}

\goalsframe

\end{document}

