
\documentclass{beamer}
\usecolortheme{dove}
\setbeamertemplate{navigation symbols}{}
\usepackage{amsmath,amssymb,amsfonts,amsthm, multicol, subfigure, color}
\usepackage{bm}
\usepackage{graphicx}
\usepackage{tabularx}
\usepackage{booktabs}
\usepackage{hyperref}
\usepackage{pdfpages}
\usepackage{xcolor}
\definecolor{seagreen}{RGB}{46, 139, 87}
\def\independenT#1#2{\mathrel{\rlap{$#1#2$}\mkern2mu{#1#2}}}
\newcommand\indep{\protect\mathpalette{\protect\independenT}{\perp}}
\def\log{\text{log}}
\newcommand\logit{\text{logit}}
\newcommand\iid{\stackrel{\text{iid}}{\sim}}
\newcommand\E{\text{E}}
\newcommand\V{\text{V}}
\renewcommand\P{\text{P}}
\newcommand{\Cov}{\text{Cov}}
\newcommand{\Cor}{\text{Cor}}
\newcommand\doop{\texttt{do}}
\usepackage{stackrel}
\usepackage{tikz}
\usetikzlibrary{arrows,shapes.arrows,positioning,shapes,patterns,calc}
\newcommand\slideref[1]{\vskip .1cm \tiny \textcolor{gray}{{#1}}}
\newcommand\red[1]{\color{red}#1}
\newcommand\blue[1]{\color{blue}#1}
\newcommand\gray[1]{\color{gray}#1}
\newcommand\seagreen[1]{\color{seagreen}#1}
\newcommand\purple[1]{\color{purple}#1}
\newcommand\orange[1]{\color{orange}#1}
\newcommand\black[1]{\color{black}#1}
\newcommand\white[1]{\color{white}#1}
\newcommand\teal[1]{\color{teal}#1}
\newcommand\magenta[1]{\color{magenta}#1}
\newcommand\Fuchsia[1]{\color{Fuchsia}#1}
\newcommand\BlueGreen[1]{\color{BlueGreen}#1}
\newcommand\bblue[1]{\textcolor{blue}{\textbf{#1}}}
\newcommand\bred[1]{\textcolor{red}{\textbf{#1}}}
\newcommand\bgray[1]{\textcolor{gray}{\textbf{#1}}}
\newcommand\bgreen[1]{\textcolor{seagreen}{\textbf{#1}}}
\newcommand\bref[2]{\href{#1}{\color{blue}{#2}}}
\colorlet{lightgray}{gray!40}
\pgfdeclarelayer{bg}    % declare background layer for tikz
\pgfsetlayers{bg,main} % order layers for tikz
\newcommand\mycite[1]{\begin{scriptsize}\textcolor{darkgray}{(#1)}\end{scriptsize}}
\newcommand{\tcframe}{\frame{
%\small{
\only<1|handout:0>{\tableofcontents}
\only<2|handout:1>{\tableofcontents[currentsubsection]}}
%}
}

\newcommand{\goalsframe}{\begin{frame}{Learning goals for today}
By the end of class, you will be able to
\begin{itemize}
    \item argue normatively about what society ought to look like
    \item outline one definition of justice
    \item apply that definition to make us care about evidence that may come from data
    %\item the original position
    %\item the equality principle
    %\item the difference principle
\end{itemize} \vskip .2in
\end{frame}}

\usepackage[round]{natbib}
\bibliographystyle{humannat-mod}
\setbeamertemplate{enumerate items}[default]
\usepackage{mathtools}

\title{Studying Social Inequality with Data Science}
\author{Ian Lundberg}
\date{\today}

\begin{document}

\section{Visualize a Distribution}

\section{Summary Statistic}

\section{How to Choose}

\begin{frame}

The best summary statistic depends on 

\end{frame}

\begin{frame}{The original position}{Rawls p.~12} \pause

Imagine a setting where:
\begin{itemize}
\item ``no one knows his place in society'' \pause
\item ``his class position or social status'' \pause
\item ``his fortune in the distribution of natural assets and abilities'' \pause
\item ``his intelligence, strength, and the like'' \pause
\end{itemize} \vskip .3in

What principles would we choose?
%``Since...no one is able to design principles to favor his particular condition, the principles of justice are the result of a fair agreement''

\end{frame}

%\begin{frame}{Two principles chosen in the original position}

%\begin{enumerate}
%\item Equality principle
%\item Difference principle
%\end{enumerate}

%\end{frame}

\begin{frame}
would we allow slavery?
\end{frame}

\begin{frame}{First principle: Equality of liberty}

%``equality in the assignment of basic rights and duties''\vskip .1in --- Rawls p.~14
``each person is to have an equal right to the most extensive basic liberty compatible with a similar liberty for others'' \vskip .1in
--- Rawls p.~60

\end{frame}

\begin{frame}
would we require complete equality of outcomes?
\end{frame}


\begin{frame}{A possible claim} \pause
if admission to Cornell depends\\only on talent to succeed at Cornell,\vskip .1in
then society is just for this outcome
\end{frame}

\begin{frame}{A possible claim}
if admission to Cornell depends\\only on talent to succeed at Cornell,\vskip .1in and we modify K--12 education to\\equally develop all children’s talents,\vskip .1in then society is just for this outcome
\end{frame}

\begin{frame}{Taken to its extreme: Liberal equality}
Children born with better talents will be admitted to Cornell.\\Others will not. \vskip .2in
Is that just?
\end{frame}

\begin{frame}{The problem with liberal equality} \pause

The luck of social circumstances are abitrary.\\ \pause
The luck of natural ability is also arbitrary. \vskip .2in \pause

``For once we are troubled by the influence of either social contingencies or natural chance on the determination of distributive shares, we are bound, on reflection, to be bothered by the influence of the other. \vskip .1in From a moral standpoint the two seem to be equally arbitrary.'' \vskip .2in
\hfill (Rawls p.~75)

\end{frame}

\begin{frame}

how can we allow any departure from equality?

\end{frame}

\begin{frame}{Second principle: Difference principle}

``social and economic inequalities, for example inequalities of wealth and authority, are just only if they result in compensating benefits for everyone, and in particular for the least advantaged members of society''\vskip .1in --- Rawls p.~14--15 \vskip .3in \pause
Example implications: \pause
\begin{itemize}
\item Admit the talented people to Cornell\\to the degree that it helps the less talented \pause
\item Raise CEO pay only if\\that will reduce poverty
\end{itemize}

%``social and economic inequalities are\\to be arranged so that they are both\\(a) reasonably expected to be to everyone's advantage and\\(b) attached to positions and offices open to all''\vskip .1in --- Rawls p.~60

\end{frame}

\begin{frame}{Two principles of justice}{according to Rawls}

A just society is the one we would choose\\if we did not know our place in society \vskip .3in

Rawls thinks we would choose two principles:
\begin{enumerate}
\item maximal liberty compatible with similar liberty for others
\item inequality only if it benefits everyone
\end{enumerate} %\vskip .3in \pause
%Once we choose this normative definition of justice,\\
%data science can reveal evidence of injustice

\end{frame}

%\begin{frame}{Recap: Two principles of justice}

%From the original position, Rawls thinks we would choose \vskip .1in
%\begin{enumerate}
%\item \bgray{Equality of liberty:} ``each person is to have an equal right to the most extensive basic liberty compatible with a similar liberty for others'' \vskip .1in
%\item \bgray{Difference principle:} ``social and economic inequalities are\\to be arranged so that they are both\\(a) reasonably expected to be to everyone's advantage and\\(b) attached to positions and offices open to all''
%\end{enumerate}

%\end{frame}

\begin{frame}{Data science can reveal evidence of injustice} \pause

With justice as defined by Rawls,\\
how have our exercises revealed injustice? \vskip .2in \pause

Exercises in the course:
\begin{itemize}
\item percentiles of the income distribution
\item salaries of baseball players
\item gender gaps in the labor market
\item predicting income given family background
\item racial residential segregation
\item racial wealth gap
\item class gaps in pay
\end{itemize}

\end{frame}

\begin{frame}{Beyond our course}

Can you think of other settings\\where the original position can\\help us decide what is just?

\end{frame}

\goalsframe

\end{document}

