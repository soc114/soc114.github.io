
\documentclass{beamer}
\usecolortheme{dove}
\setbeamertemplate{navigation symbols}{}
\usepackage{amsmath,amssymb,amsfonts,amsthm, multicol, subfigure, color}
\usepackage{bm}
\usepackage{graphicx}
\usepackage{tabularx}
\usepackage{booktabs}
\usepackage{hyperref}
\usepackage{pdfpages}
\usepackage{xcolor}
\definecolor{seagreen}{RGB}{46, 139, 87}
\def\independenT#1#2{\mathrel{\rlap{$#1#2$}\mkern2mu{#1#2}}}
\newcommand\indep{\protect\mathpalette{\protect\independenT}{\perp}}
\def\log{\text{log}}
\newcommand\logit{\text{logit}}
\newcommand\iid{\stackrel{\text{iid}}{\sim}}
\newcommand\E{\text{E}}
\newcommand\V{\text{V}}
\renewcommand\P{\text{P}}
\newcommand{\Cov}{\text{Cov}}
\newcommand{\Cor}{\text{Cor}}
\newcommand\doop{\texttt{do}}
\usepackage{stackrel}
\usepackage{tikz}
\usetikzlibrary{arrows,shapes.arrows,positioning,shapes,patterns,calc}
\newcommand\slideref[1]{\vskip .1cm \tiny \textcolor{gray}{{#1}}}
\newcommand\red[1]{\color{red}#1}
\newcommand\blue[1]{\color{blue}#1}
\newcommand\gray[1]{\color{gray}#1}
\newcommand\seagreen[1]{\color{seagreen}#1}
\newcommand\purple[1]{\color{purple}#1}
\newcommand\orange[1]{\color{orange}#1}
\newcommand\black[1]{\color{black}#1}
\newcommand\white[1]{\color{white}#1}
\newcommand\teal[1]{\color{teal}#1}
\newcommand\magenta[1]{\color{magenta}#1}
\newcommand\Fuchsia[1]{\color{Fuchsia}#1}
\newcommand\BlueGreen[1]{\color{BlueGreen}#1}
\newcommand\bblue[1]{\textcolor{blue}{\textbf{#1}}}
\newcommand\bred[1]{\textcolor{red}{\textbf{#1}}}
\newcommand\bgray[1]{\textcolor{gray}{\textbf{#1}}}
\newcommand\bgreen[1]{\textcolor{seagreen}{\textbf{#1}}}
\newcommand\bref[2]{\href{#1}{\color{blue}{#2}}}
\colorlet{lightgray}{gray!40}
\pgfdeclarelayer{bg}    % declare background layer for tikz
\pgfsetlayers{bg,main} % order layers for tikz
\newcommand\mycite[1]{\begin{scriptsize}\textcolor{darkgray}{(#1)}\end{scriptsize}}
\newcommand{\tcframe}{\frame{
%\small{
\only<1|handout:0>{\tableofcontents}
\only<2|handout:1>{\tableofcontents[currentsubsection]}}
%}
}

\newcommand{\goalsframe}{\begin{frame}{Learning goals for today}
By the end of class, you will be able to
\begin{itemize}
    \item understand the notion of supervised machine learning
    \item apply that notion to the specific case of regression trees
 \end{itemize} 
  \vskip .2in
\end{frame}}

\usepackage[round]{natbib}
\bibliographystyle{humannat-mod}
\setbeamertemplate{enumerate items}[default]
\usepackage{mathtools}

\title{Studying Social Inequality with Data Science}
\author{Ian Lundberg}
\date{\today}

\begin{document}

\begin{frame}
\begin{tikzpicture}[x = \textwidth, y = \textheight]
\node at (0,0) {};
\node at (1,1) {};
\node[anchor = north west, align = left, font = \huge] at (0,.9) {Social\\Data\\Science};
\node[anchor = north east, align = right] (number) at (1,.9) {Soc 114\\Winter 2025};
\node[anchor = north, font = \Large, align = center] at (.5,.5) {Project Rubric: Review in Lecture};
\end{tikzpicture}
\end{frame}

%\goalsframe

\begin{frame}[t]{Rubric item 1: Descriptive question}
\vskip .4in
The descriptive research question is well motivated, feasible, and avoids causal language. \vskip .2in

\begin{itemize}
\item We estimate the employment rate among those who are and are not enrolled in college, for U.S. residents ages 18--22 in 2024.
\end{itemize}

\end{frame}

\begin{frame}[t]{Rubric item 1: Descriptive question}
\vskip .4in

The descriptive research question is well motivated, feasible, and avoids causal language. \vskip .2in

Examples with problems:
\begin{itemize}
\item We estimate how college enrollment leads to less employment...
\item We estimate the relative importance of three predictors of employment...
\end{itemize} \pause \vskip .2in
Remember that a descriptive question is\\a (summary statistic) of (outcome) among (subgroups)

\end{frame}

\begin{frame}[t]{Rubric item 2: Causal question}
\vskip .4in

The causal research question is well motivated, feasible, and states appropriate causal assumptions. \vskip .2in

A high-scoring submission might involve
\begin{itemize}
\item Average causal effect of XX on YY among ZZ
\item Potential outcomes defined
\item DAG included
\item Tells us what is the sufficient adjustment set
\end{itemize}

\end{frame}

\begin{frame}[t]{Rubric items 3--7: Elements of the research question}
\vskip .4in

\begin{itemize}
\item Unit of analysis is defined clearly
\item Target population is defined clearly
\begin{itemize}
\item (Not your sample---population from which your sample is drawn)
\end{itemize}
\item Predictor(s) are defined clearly
\item Outcome is defined clearly
\item Summary statistic is defined clearly (e.g., mean, median, proportion)
\end{itemize}

\end{frame}

\begin{frame}[t]{Rubric items 8--10: Data}
\vskip .4in

\begin{itemize}
\item Data source is described
\item Sample restrictions are precisely stated, with the number of cases dropped at each step
\item Weights are used appropriately, if applicable
\end{itemize}

\end{frame}

\begin{frame}[t]{Rubric items 11--14: Code style and presentation of results}
\vskip .4in

\begin{itemize}
\item Text explains and interprets the figure(s)
\item Writing is concise and grammatically correct
\item Code is well-formatted, well-documented, and easy to follow
\item Code lines fit on the PDF page
\end{itemize}

\end{frame}

%\goalsframe

\end{document}
